\chapter{Background Estimate }

\section{Monte Carlo }
{\normalsize - MC's used, PDF's, k-factors}

\section{Fake Factor Multi-jet estimate }

One of the major sources of background to di-electron signals are di-jets or electron+jets (mainly W+jets) events where one or both selected leptons are jets faking electron signatures. The method for estimating this background, described here, is a "fake factor" or "matrix-method". This is a data driven method where electrons are selected by a tight ($N_{tight}$) and loose ($N_{loose}$) selection. The tight selection is the standard electron selection used in this analysis while the loose selection has no isolation requirement and must only pass a loose++ egamma definition with no track matching criteria. $N_{tight}$ is therefore by design a subset of $N_{loose}$. Two more hidden values are also assigned $real$ and $fake$ referring to true source of each electron. This gives us two coefficients to determine from data.

\begin{equation} \label{eq:fakeRate}
   f~=~\frac{N^{fake}_{tight}}{N^{fake}_{loose}} \qquad \qquad r~=~\frac{N^{real}_{tight}}{N^{real}_{loose}}
\end{equation}

The fake rate $f$ denotes the probability that a $fake$ electron which passes the loose requirement also passes tight while $r$ refers to the probability that a $real$ electron which passes the loose requirement also passes the tight.
Reconstructed events are split in to two distinct groups, tight($T$) and loose while failing tight($L$) where $Tight$ is now no longer a subset of $Loose$. This allows us to relate our reconstructed events to the underling truth events via a matrix of fake rates Eq.~\ref{eq:mainFakeMatrix}.

\begin{equation} \label{eq:mainFakeMatrix}
   \begin{pmatrix}
      N_{TT} \\
      N_{TL} \\
      N_{LT} \\
      N_{LL} \\
   \end{pmatrix}
   =
   \begin{pmatrix}
      r_{1}r_{2} & r_{1}f_{2} & f_{1}r_{2} & f_{1}f_{2} \\
      r_{1}(1-r_{2}) & r_{1}(1-f_{2}) & f_{1}(1-r_{2}) & f_{1}(1-f_{2}) \\
      (1-r_{1})r_{2} & (1-r_{1})f_{2} & (1-f_{1})r_{2} & (1-f_{1})f_{2} \\
      (1-r_{1})(1-r_{2}) & (1-r_{1})(1-f_{2}) & (1-f_{1})(1-r_{2}) & (1-f_{1})(1-f_{2}) \\
   \end{pmatrix}
   \begin{pmatrix}
      N_{RR} \\
      N_{RF} \\
      N_{FR} \\
      N_{FF} \\
   \end{pmatrix}
\end{equation}

The first index refers to the highest $p_{T}$ electron while the second index refers to the second highest $p_{T}$ electron. So $N_{LT}$ indicates the reconstructed events with highest $p_{T}$ electron only passing the $Loose$ selection while the second highest $p_{T}$ electron passes $Tight$ selection. The indexes 1 and 2 refer to fake rates ($f$) and efficiencies ($r$) on leading and sub-leading electrons respectively.

The interesting part for this study is contribution to $N_{TT}$ coming from sources other $N_{RR}$, these can be seen in Eq.~\ref{eq:multijet}.

\begin{align} \label{eq:multijet}
   N^{\ell+jets}_{TT}~&=~r_{1}f_{2}N_{RF}~+~f_{1}r_{2}N_{FR} \nonumber \\
   N^{di-jets}_{TT}~&=~f_{1}f_{2}N_{FF} \nonumber \\
   N^{\ell+jets~\&~di-jets}_{TT}~&=~r_{1}f_{2}N_{RF}~+~f_{1}r_{2}N_{FR}~+~f_{1}f_{2}N_{FF} 
\end{align}

This function however contains hidden variables and so Eq.~\ref{eq:mainFakeMatrix} is inverted to derive a better formalism.

\begin{equation}
   \begin{pmatrix}
      N_{RR} \\
      N_{RF} \\
      N_{FR} \\
      N_{FF} \\
   \end{pmatrix}
   = \alpha
   \begin{pmatrix}
      (f_{1}-1)(f_{2}-1) & (f_{1}-1)f_{2} & f_{1}(f_{2}-1) & f_{1}f_{2} \\
      (f_{1}-1)(1-r_{2}) & (1-f_{1})r_{2} & f_{1}(1-r_{2}) & -f_{1}r_{2} \\
      (r_{1}-1)(1-f_{2}) & (1-r_{1})f_{2} & r_{1}(1-f_{2}) & -r_{1}f_{2} \\
      (1-r_{1})(1-r_{2}) & (r_{1}-1)r_{2} & r_{1}(r_{2}-1) & r_{1}r_{2} \\
   \end{pmatrix}
   \begin{pmatrix}
      N_{TT} \\
      N_{TL} \\
      N_{LT} \\
      N_{LL} \\
   \end{pmatrix}
\end{equation}

where,

\begin{equation}
   \alpha~=~\frac{1}{(r_{1}-f_{1})(r_{2}-f_{2})}
\end{equation}

The fraction of selected events with at least one fake is then given by,

\begin{equation}
\begin{aligned}
   N^{\ell+jets~\&~di-jets}_{TT}~&=&~\alpha r_{1}f_{2}[(f_{1}-1)(1-r_{2})N_{TT}~+~(1-f_{1})r_{2}N_{TL}~&+~f_{1}(1-r_{2})N_{LT}~-~f_{1}r_{2}N_{LL}] \\
      &~+&~\alpha f_{1}r_{2}[(r_{1}-1)(1-f_{2})N_{TT}~+~(1-r_{1})f_{2}N_{TL}~&+~r_{1}(1-f_{2})N_{LT}~-~r_{1}f_{2}N_{LL}] \\
      &~+&~\alpha f_{1}f_{2}[(1-r_{1})(1-r_{2})N_{TT}~+~(r_{1}-1)r_{2}N_{TL}~&+~r_{1}(r_{2}-1)N_{LT}~+~r_{1}r_{2}N_{LL}] 
\end{aligned}
\end{equation}

\begin{equation} \label{eq:mainFakeResult}
\begin{aligned}
   =\alpha[r_{1}f_{2}(f_{1}-1)(1-r_{2})~+~f_{1}r_{2}(r_{1}-1)(1-f_{2})~+~f_{1}f_{2}(1-r_{1})(1-r_{2})]N_{TT} \\
   +~\alpha f_{2}r_{2}[r_{1}(1-f_{1})~+~f_{1}(1-r_{1})~+~f_{1}(r_{1}-1)]N_{TL} \\
   +~\alpha f_{1}r_{1}[f_{2}(1-r_{2})~+~r_{2}(1-f_{2})~+~f_{2}(r_{2}-1)]N_{LT} \\
   -~\alpha f_{1}f_{2}r_{1}r_{2}N_{LL}
\end{aligned}
\end{equation}

Equation ~\ref{eq:mainFakeResult} shows the derived formula relating the multi-jet background to fake rates, efficiencies and four independent samples selected from data. Detailed here is this method used on the full $20~fb^{-1}$ of integrated luminosity from ATLAS's 2012 run.


\subsection{Real electron efficiency estimation}

The real electron efficiency is defined as Eq. ~\ref{eq:fakeRate} $r~=~N^{real}_{tight}/N^{real}_{loose}$. This is determined from MC using a mass binned Drell-Yan sample. The efficiencies are found for both the leading and sub-leading electrons and binned in 8 $p_{T}$ and three eta bins of $|\eta|<1.37$ (barrel), $1.52<|\eta|<2.01$ and $2.01<|\eta|<2.47$ (endcap). The efficiency is distributed between $90$ - $96\%$.\\
\\
--plots of Real electron efficiencies.


\subsection{Fake electron rate estimation}

The default method selected for analysing the fake rates is a single object method on the jet stream. This gives the main advantage of more statistics and a higher energy reach compared to methods such as using tag and probe in the egamma stream.
An array of triggers are used for selecting suitable events based on the single jet trigger EF\_jX\_a4chad (where X = 25, 35, 45, 55, 80, 110, 145, 180, 220, 280, 360). Events are associated to groups with the lowest trigger threshold they pass as each trigger has a different prescale. Objects are selected with the AntiKt4TopoEMJets algorithm and then matched to objects in the egamma stream with a $\Delta R~<~0.1$. Objects also have to pass the medium jet-cleaning criteria. Two ferther steps are taken to suppress real electrons from W decays and real Drell-Yan events. A veto of $E_{Tmiss}~>~25~GeV$ is introduced to combat the former while events with two medium++ or loose++ electrons with $|m_{tag~\&~probe}-91~GeV|~<~20~GeV$ are vetoed to counter the real Drell-Yan.

The fake rate is then defined as Eq. ~\ref{eq:fakeRate} $r~=~N^{fake}_{tight}/N^{fake}_{loose}$ with distributions selected using the standard event selection on the matched egamma objects.
Due to the different prescales of each trigger a separate set of fake rates are calculated for each trigger, these are then combined as a weighted average of all fake rates.
\\
\\
--plots of fake rates. (describe how these are binned, slightly different to efficiencies)\\

\subsection{Properties of Multi-jet background}

In order to compose the final sample you take these efficiencies and fake rates and using Eq. ~\ref{eq:mainFakeResult} you select each event as either $N_{TT}$, $N_{TL}$, $N_{LT}$ or $N_{LL}$ and apply its weight according to each electrons $p_{T}$, $\eta$ and therefore corresponding fake rate and efficiency.
\\
\\
-- plots showing individual $N_{TT}$, $N_{TL}$, $N_{LT}$ and $N_{LL}$ parts, as well as full composed background.\\
\\
-- fit around Z peak due to unreliable modelling.\\

\subsection{Other methods and estimation of Error}

-- tag and probe on jet stream\\
\\
-- tag and probe on jet stream\\
\\
-- Flat $20\%$ Systematic is used.\\










