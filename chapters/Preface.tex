\chapter*{Preface}
\addcontentsline{toc}{chapter}{Preface}

This thesis describes the work carried out for an analysis searching for new non-resonant physics with the ATLAS detector. The thesis focuses on the search within the electron decay channel using ATLAS's 8 TeV data set. This is compared and contrasted with the previous ATLAS search using the 7 TeV data set in chapter 5 showing the evolution of this analysis. The author made a major contribution to these two analyses and the resulting papers \cite{PhysRevD.87.015010} as well as working as part of the groups looking at new resonant physics and contributing to two other papers \cite{}. The author also made a major contribution to work in the electron photon triggering group detailed in section \ref{sec:TrigRates}. This work composed part of an atlas note \cite{} with the author presenting a related poster at the Computing High Energy Particle physics conference (CHEP) in 2012. 
The authors service task on ATLAS was composed of this work and maintenance of the high level trigger code.
The 8 TeV non-resonant analysis discussed in this thesis was primarily carried out within a group of four students, one researcher and four academics working on ATLAS. The search within the electron channel was primarily carried out by two students with the author focusing on the Contact Interaction model and necessarily this dictates the focus on the Contact interaction model within this thesis. For the 7 TeV analysis the author was the only electron channel analyst. The search for both models complemented each other strongly and so therefore and discussion of both is seen as important. 


% Trigger work:\\

% - optimisation of 2012 electron triggers at LV1 up to HLT for higher luminoscity conditions and the required rate reductions.\\

% Z Prime Analysis:\\

% - Ran analysis of several MC sample for people. (including Black holes, pythia DY and samples for \\

% Section \ref{sec:TrigRates}
% There was also a contribution to the maintenance of the $e/\gamma$ trigger software run in the ATLAS detector, both these tasks forming the authorship qualification. The authorship task culminated in presentation of a poster on behalf of the $e/\gamma$ trigger group at the Computing and High Energy Physics (CHEP) conference held New York in May 2012. 



% Non-resonant Analysis:\\

% - Full analysis coded and run by liam.\\

% - production of reverse ID jets sample for Non-resonant anlysis.\\

% - optimisation of new isolation cut.\\

% - Study of new opposite sign cut and effect on reverse ID jets sample.\\

% - Study of cosTheta* variable data MC comparison in control region.\\

% - optimisation of binning for statistical analysis.\\

% - Limit setting via Baysian analysis toolkit.\\






Following is an overview of this thesis describing the contents each chapter. Chapters 1 to 4 contain background to the theory and the ATLAS experiment and are not work carried out by the author apart from a section on trigger rates at high luminosity referenced above. While chapters 5 to 10 detail analysis work carried out by the author where related work not completed by the author has been indicated. The thesis is followed by an appendix containing additional material and information not contained in the body of the thesis.

\begin{itemize}
\item{ 
{\bf Chapter 1: Theory} \\
This chapter covers an overview of the Standard Model (SM) of particle physics and then continue on to Beyond the Standard model (BSM) phenomena. The main focus is on the idea of Non resonant excesses in the dilepton Drell-Yan (DY) spectrum of which two examples are discussed. The first example is Contact Interactions, a model which describes many BSM phenomena that can show as four fermion contact interaction that exhibit a divergence from the SM DY spectrum. The example shown is that of a quark-lepton composite model where at a certain energy level quarks and leptons can form composite particles. The second example given is the Arkani-Hamed, Dimopoulos, and Dvali (ADD) model. This is a graviton theory with the addition of large extra spacial dimensions to dilute gravity. These large extra spacial dimensions create Kaluza-Klein resonances of the graviton very close to each other and so exhibit signs of Non-resonance behaviour. A look at past results for similar searches is also discussed here.
}
\item{ 
{\bf Chapter 2: Experiment} \\
This chapter covers an overview of the ATLAS experiment and the LHC. A particular focus will be given to the inner tracking detector and energy Calorimeters of ATLAS as these systems are the parts used in the detection of di-electron events used in this analysis. Although parts of the detector will be discussed in some respect.
}
\item{ 
{\bf Chapter 3: Trigger} \\
This chapter focuses on the triggering system for selecting data events in the ATLAS detector. An overview of the whole system will be given but a focus made on the ``egamma'' part which selects electron and photon events. A slight detour will be made discussing the effect of increases in the luminosity of the LHC beam through the 2011-2012 data taking period and efforts taken to reduce high rates of data acquisition this entailed in the ``egamma'' chain.
}
\item{ 
{\bf Chapter 4: Event Reconstruction} \\
This chapter details the algorithms used in reconstructing electrons and photons from the detector output. It also contains a discussion on ATLAS assignments of $tight$, $medium$ and $loose$ electrons.
}
\item{ 
{\bf Chapter 5: Non-Resonance 7 TeV Analysis} \\
This chapter looks at the first non-resonant analysis completed on the 7 TeV data set from 2011 with a luminosity of $4.9~fb^{-1}$. An overview of the full event selection and then limits set is included along with some comparisons between this and the following analysis. 
}
\item{ 
{\bf Chapter 6: Event Selection} \\
This chapter covers the main event selection of di-electron events for the non-resonance analysis on the $20~fb^{-1}$ recorded in 2012. There will also be a discussion of and need for corrections applied to energy measurements.
}
\item{ 
{\bf Chapter 7: Background Estimate} \\
This chapter discuses the estimate made of the background processes to the non-resonant signal. It covers the Monte Carlo (MC) generated to estimate these backgrounds as well as corrections applied to match MC to the data collection conditions used and corrections to account for next to next to leading order calculation effects.
}
\item{ 
{\bf Chapter 8: Signal Search} \\
This chapter shows the search for new physics in the data collected in the 2012 data taking period. This includes a description of the MC used to predict the signals as well as comparison between the Data and the MC prediction of the background. Also looked at are the significance or p-value of any divergences from the SM background prediction.
}
\item{ 
{\bf Chapter 9: Statistical Analysis} \\
This chapter discuses a statistical treatment of the results. First discussed is possible sources of systematic error in the analysis as well as levels of statistical error. Next a Bayesian approach is taken to searching for signs of new physics and then setting lower limits on the scale of new physics predicted by this analysis.
}
\item{ 
{\bf Chapter 10: Conclusion} \\
This final chapter discuses the conclusions obtained from this analysis with an overview of the results and a look forward to the future of searches of non-resonant physics within ATLAS.
}
\end{itemize}






