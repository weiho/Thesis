\chapter*{Preface}
\addcontentsline{toc}{chapter}{Preface}

\begin{itemize}

\item{ 
{\bf Chapter 1: Theory} \\
This chapter covers an overview of the Standard Model (SM) of particle physics and then continue on to Beyond the Standard model (BSM) phenomena. The main focus is on the idea of Non resonant excesses in the dilepton Drell-Yan (DY) spectrum of which two examples are discussed. The first example is Contact Interactions, a model which describes many BSM phenomena that can show as four fermion contact interaction that exhibit a divergence from the SM DY spectrum. The example shown is that of a quark-lepton composite model where at a certain energy level quarks and leptons can form composite particles. The second example given is the Arkani-Hamed, Dimopoulos, and Dvali (ADD) model. This is a graviton theory with the addition of large extra spacial dimensions to dilute gravity. These large extra spacial dimensions create Kaluza-Klein resonances of the graviton very close to each other and so exhibit signs of Non-resonance behaviour. A look at past results for similar searches is also discused here.
}
\item{ 
{\bf Chapter 2: Experiment} \\
This chapter covers an overview of the ATLAS experiment and the LHC. A particular focus will be given to the inner tracking detector and energy Calorimeters of ATLAS as these systems are the parts used in the detection of di-electron events used in this analysis. Although parts of the detector will be discussed in some respect.
}
\item{ 
{\bf Chapter 3: Trigger} \\
This chapter focuses on the triggering system for selecting data events in the ATLAS detector. An overview of the whole system will be given but a focus made on the "egamma" part which selects electron and photon events. A slight detour will be made discussing the effect of increases in the luminosity of the LHC beam through the 2011-2012 data taking period and efforts taken to reduce high rates of data acquisition this entailed in the "egamma" chain.
}
\item{ 
{\bf Chapter 4: Reconstruction} \\
This chapter details the algorithms used in reconstructing electrons and photons from the detector output. It also contains a discussion on ATLAS assignments of $tight$, $medium$ and $loose$ electrons.
}
\item{ 
{\bf Chapter 5: Event Selection} \\
This chapter covers the main event selection of di-electron events for the non-resonance analysis on the $20~fb^{-1}$ recorded in 2012. There will also be a discussion of and need for corrections applied to energy measurements.
}
\item{ 
{\bf Chapter 6: Background Estimate} \\
This chapter discuses the estimate made of the background processes to the non-resonant signal. It covers the Monte Carlo (MC) generated to estimate these backgrounds as well as corrections applied to match MC to the data collection conditions used and corrections to account for next to next to leading order calculation effects.
}
\item{ 
{\bf Chapter 7: Signal Search} \\
This chapter shows the search for new physics in the data collected in the 2012 data taking period. This includes a description of the MC used to predict the signals as well as comparison between the Data and the MC prediction of the background. Also looked at are the significance or p-value of any divergences from the SM background prediction.
}
\item{ 
{\bf Chapter 8: Statistical Analysis and Conclusion} \\
This final chapter discuses a statistical treatment of the results. First discussed is possible sources of systematic error in the analysis as well as levels of statistical error. Next a Bayesian approach is taken to setting lower limits on the scale of new physics predicted by this analysis. Last is a discussion of how this result can be interpreted and a comparison to past searches for similar phenomena and current ones. $A_{\mu}$
}
\end{itemize}




Trigger work:\\

- optimisation of 2012 electron triggers at LV1 up to HLT for higher luminoscity conditions and the required rate reductions.\\

Z Prime Analysis:\\

- Ran analysis of several MC sample for people. (including Black holes, pythia DY and samples for \\


Non-resonant Analysis:\\

- Full analysis coded and run by liam.\\

- production of reverse ID jets sample for Non-resonant anlysis.\\

- optimisation of new isolation cut.\\

- Study of new opposite sign cut and effect on reverse ID jets sample.\\

- Study of cosTheta* variable data MC comparison in control region.\\

- optimisation of binning for statistical analysis.\\

- Limit setting via Baysian analysis toolkit.\\


