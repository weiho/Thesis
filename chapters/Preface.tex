\chapter*{Preface}
\addcontentsline{toc}{chapter}{Preface}
\pagestyle{headings}
\pagestyle{headings}

This thesis describes the work carried out for an analysis searching for new non-resonant physics with the ATLAS detector. It focuses on the search within the electron decay channel using ATLAS's 8 TeV data set. This is compared and contrasted with the previous ATLAS search using the 7 TeV data set in chapter \ref{ch:7tev} showing the evolution from this previous analysis. For the 7 TeV analysis the author was the only electron channel analyst while they were one of two analysts for the 8 TeV analysis. The author made a major contribution to these two analyses that are detailed in the 7 TeV \cite{PhysRevD.87.015010} and 8 TeV \cite{ATLAS-CONF-2014-030} publications.
The 8 TeV non-resonant analysis discussed in this thesis was primarily carried out within a group of four students, one researcher and four academics working on ATLAS. The search within the electron channel was primarily carried out by two students with the author focusing on the Contact Interaction model and necessarily this dictates the focus on this model within this thesis. Both theoretical models (Contact Interactions and ADD) are presented however as the search methods strongly complement each other and the comparison is seen as important.
The author also aided in another analysis contributing to a 7 TeV \cite{Aad:2012hf} and an 8 TeV \cite{Z:1515998} publication searching for new resonant physics in the dilepton channel. This resonant analysis has strong ties with the analysis presented here and the author worked on the dielectron analysis for both.
The author also made a major contribution to work in the electron photon triggering group detailed in section \ref{sec:TrigRates}. This work composed part of an ATLAS note \cite{ATL-PHYS-PUB-2011-007} with the author presenting a related poster \cite{Duguid:1450151} at the Computing High Energy Particle physics conference (CHEP) in 2012 in New York. 
The author's service task on ATLAS was composed of this work and maintenance of the associated high level trigger code.



% Trigger work:\\

% - optimisation of 2012 electron triggers at LV1 up to HLT for higher luminoscity conditions and the required rate reductions.\\

% Z Prime Analysis:\\

% - Ran analysis of several MC sample for people. (including Black holes, pythia DY and samples for \\

% Section \ref{sec:TrigRates}
% There was also a contribution to the maintenance of the $e/\gamma$ trigger software run in the ATLAS detector, both these tasks forming the authorship qualification. The authorship task culminated in presentation of a poster on behalf of the $e/\gamma$ trigger group at the Computing and High Energy Physics (CHEP) conference held New York in May 2012. 



% Non-resonant Analysis:\\

% - Full analysis coded and run by liam.\\

% - production of reverse ID jets sample for Non-resonant anlysis.\\

% - optimisation of new isolation cut.\\

% - Study of new opposite sign cut and effect on reverse ID jets sample.\\

% - Study of cosTheta* variable data MC comparison in control region.\\

% - optimisation of binning for statistical analysis.\\

% - Limit setting via Baysian analysis toolkit.\\






Following is an overview of this thesis describing the contents of each chapter. Chapters 1 to 4 contain background to the theory and the ATLAS experiment and do not contain work carried out by the author apart from a section on trigger rates at high luminosity referenced above. Chapters 5 to 11 detail analysis work carried out by the author where related work not completed by the author has been indicated. The thesis is followed by an appendix containing additional material and information not contained in the body of the thesis.

\begin{itemize}
\item{ 
{\bf Chapter 1: Theory} \\
This chapter covers an overview of the Standard Model (SM) of particle physics and then continues on to Beyond the Standard Model (BSM) phenomena. The main focus is on the idea of non-resonant excesses in the dilepton Drell-Yan (DY) spectrum of which two examples are discussed. The first example is Contact Interactions, a model which describes many BSM phenomena that can show as four fermion contact interaction that exhibit a divergence from the SM DY spectrum. The example shown is that of a quark-lepton composite model where quarks and leptons are found to be composed of smaller particles. The second example given is the Arkani-Hamed, Dimopoulos, and Dvali (ADD) model. This is a Graviton theory with the addition of large extra spacial dimensions to dilute gravity. These large extra spacial dimensions create Kaluza-Klein resonances of the graviton very close to each other and so exhibit signs of non-resonance behaviour. A look at past results for similar searches is also discussed here.
}
\item{ 
{\bf Chapter 2: Experiment} \\
This chapter is an overview of the ATLAS experiment and the LHC with important detector and LHC components discussed. A particular focus is given to the inner tracking detector and energy calorimeters of ATLAS as these systems are the parts used in the detection of di-electron events used in this analysis.
}
\item{ 
{\bf Chapter 3: The Trigger \& Data Acquisition} \\
This chapter focuses on the triggering system for selecting data events in the ATLAS detector. An overview of the whole system will be given but a focus made on the ``egamma'' trigger which selects electron and photon events. A slight detour will be made discussing the effect of increases in the luminosity of the LHC collisions through the 2011-2012 data-taking period and efforts taken to reduce high rates of data acquisition this led to in the ``egamma'' chain.
}
\item{ 
{\bf Chapter 4: Event Reconstruction} \\
This chapter details the algorithms used in reconstructing electrons and photons from the detector output. It also contains a discussion on ATLAS assignments of $tight$, $medium$ and $loose$ electrons.
}
\item{ 
{\bf Chapter 5: Event Selection} \\
This chapter covers the main event selection of di-electron events for the non-resonance analysis using the $20~fb^{-1}$ recorded in 2012. There is also a discussion of the necessary corrections applied to energy measurements.
}
\item{ 
{\bf Chapter 6: Background Estimate} \\
This chapter discusses the estimate of the background processes to the non-resonant signal. It covers the Monte Carlo (MC) samples generated to estimate these backgrounds as well as corrections applied to match MC to the data collection conditions used and corrections to account for next to next to leading order generator effects.
}
\item{ 
{\bf Chapter 7: Signal and Results} \\
This chapter shows the search for new physics in the data collected in the 2012 data taking period. This includes a description of the MC used to predict the signals as well as comparison between the Data and the MC prediction of the background. Also looked at are the significance or $p$-value of any divergences from the SM background prediction.
}
\item{ 
{\bf Chapter 8: Statistical Analysis} \\
This chapter discusses the statistical treatment of the results. First discussed is possible sources of systematic error in the analysis as well as levels of statistical error. Then there is a look at the complications introduced with the angular analysis in $\cos\theta^{*}$. Last a Bayesian approach is taken to search for signs of new physics and then setting lower limits on the scale of new physics predicted by this analysis.
}
\item{ 
{\bf Chapter 9: Non-Resonance 7 TeV Analysis} \\
This chapter looks at the first non-resonant analysis completed on the 7 TeV data set from 2011 with a luminosity of $4.9~fb^{-1}$. An overview of the full event selection and limits set are included along with some comparisons between this and the 8 TeV analysis. This analysis is presented after the 8 TeV analysis as a comparison to show the evolution of the analysis and highlight the authors work.
}
\item{ 
{\bf Chapter 10: Conclusion} \\
This final chapter discusses the conclusions obtained from this analysis with an overview of the results and a comparison between the limits set and previous results. Finally there is a look forward to the future of searches of non-resonant physics within ATLAS and the LHC.
}
\end{itemize}






