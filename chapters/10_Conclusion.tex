\chapter{Conclusion}

To conclude this analysis sees no evidence of new non-resonant physics at high mass in the dielectron decay channel and along with the dimuon decay channel limits are set using a Bayesian statistical approach on the scale of new physics in the dilepton decay channel for two models of non-resonant new physics, CI and ADD, for many formalisms each. The limits set mark the highest limits found for $qq\ell\ell$ contact interactions and the first limits set on some of the formalisms. While the ADD limits mark a large increase in the previous dilepton searches at the LHC.

Comparing to the previous CI ATLAS analysis \cite{PhysRevD.87.015010} where limits of $\Lambda$ $>$ 12.7 TeV and $\Lambda$ $>$ 9.63 TeV for the dilepton LL CI model for constructive and destructive interference were set the limits found here of $\Lambda$ $>$ 21.55 TeV and $\Lambda$ $>$ 19.61 TeV for the dilepton LL CI model for constructive and destructive interference mark a significant increase in these limits. Higher limits were also set for new formalism searched for, CI LR, where limits of $\Lambda$ $>$ 26.25 TeV and $\Lambda$ $>$ 23.77 TeV for the dilepton LL CI model for constructive and destructive interference were set due to the added information coming from the new angular analysis used in this analysis. As expected similar limits were set for the RR formalism as for the LL formalism due to the symmetry of these interactions. Observed limits within the electron channel were found to vary up slightly from the expected value due to a slight deficit of observed events in the 1200-1800 GeV search bin for forward and backwards. This deficit was found to not be significant and figure \ref{fig:Theta_CI_main} shows the observed limits to be in agreement with the distribution of expected limits from pseudo experiments.

ADD limits also saw a significant increase from the previous analysis \cite{PhysRevD.87.015010} with combined dilepton limits for the GRW formalism at M$_{s}$ $>$ 4.79 TeV compared to the previous limits of M$_{s}$ $>$ 2.94 TeV set on 2011 data. Limits were also converted in to many different formalisms seen in table \ref{tab:ADD_results_formalisms}. 

For completeness all limits are also calculated for two separate priors in Bayesian statistical analysis motivated by the form of the differential cross-section of new physics.


\section{Looking Forward}

Beyond this analysis ATLAS looks towards RunII due to start in 2015. With no new physics discovered beyond the standard model so far the next few years will be important for searches such as this. An increase in centre of mass energy to 13 and then to 14 TeV gains further increases in the reach of limits but after just over a year of running at the proposed centre of mass energy a maximum will be reached. This analysis is not limited by statistics but highly dependent on the energy of collisions. If new physics such as proposed here is not found within a few years of running then it will be ruled out in this form from the reach of the LHC. Non-resonant physics as well a resonant decays, particularly in the clean dilepton channel, will be some of the first physics to be ruled out or found at a new collision energy but this doesn't mean there are not other beyond the standard model process that could yet exist in nature. The standard model has been shown to make very accurate predictions for a host of phenomena yet we know it is not a complete theory. If it holds up within the energy range of the LHC is yet to be seen. 


