\chapter{Conclusion}

To conclude this analysis sees no evidence of new non-resonant physics at high mass in the dielectron decay channel and along with the dimuon decay channel sees no evidence in the dilepton decay channel either. Limits are set using a Bayesian statistical approach on the scale of new physics in the dilepton decay channel for two models of non-resonant new physics, CI and ADD. The limits set the highest limits found for $qq\ell\ell$ contact interactions with limits set on three formalisms of LL, LR and RR Contact Interactions. While the ADD limits mark a large increase in the previous dilepton searches at the LHC.

Comparing to the previous CI ATLAS analysis \cite{PhysRevD.87.015010}, detailed in chapter \ref{ch:7tev}, where limits of $\Lambda$ $>$ 12.7 TeV and $\Lambda$ $>$ 9.63 TeV for the dilepton LL CI model for constructive and destructive interference were set, limits established using the 8 TeV data of $\Lambda$ $>$ 21.55 TeV and $\Lambda$ $>$ 19.61 TeV for the dilepton LL CI model for constructive and destructive interference mark a significant increase. Higher limits were also set for new LR formalism where limits of $\Lambda$ $>$ 26.25 TeV and $\Lambda$ $>$ 23.77 TeV for constructive and destructive interference were obtained which was made possible by the additional information coming from the new angular analysis used. As expected similar limits were set for the RR formalism as for the LL formalism due to the symmetry of these interactions. Observed limits within the electron channel were found to vary up slightly from the expected value due to a slight deficit of observed events in the 1200-1800 GeV search bin for forward and backwards. This deficit was found to not be significant and figure \ref{fig:Theta_CI_main} shows the observed limits to be in agreement with the distribution of expected limits from pseudo experiments.

CMS have not yet released limits on contact interactions for the 8 TeV data set to compare to.

ADD limits also saw a significant increase from the previous analysis \cite{PhysRevD.87.015010} with combined dilepton limits for the GRW formalism at M$_{s}$ $>$ 4.79 TeV compared to the previous limits of M$_{s}$ $>$ 2.94 TeV set on 2011 data. Limits were also converted in to many different formalisms seen in table \ref{tab:ADD_results_formalisms}. 

For completeness all limits are also calculated for two separate priors in Bayesian statistical analysis motivated by the form of the differential cross-section of new physics for both CI and ADD.


\section{Looking Forward}

Beyond this analysis, ATLAS looks towards RunII due to start in 2015. With no new physics discovered beyond the Standard Model so far, the next few years will be important for searches such as this. An increase in centre of mass energy to 13 and eventually 14 TeV would give an increase in the reach of limits but after just over a year of running at the proposed centre of mass energy a final limit for the LHC will be reached. The final limit is because centre of mass energy is far more important in the sensitivity of the non resonant search than statistics obtained from a higher integrated luminosity. If new physics such as proposed here is not found within a few years of running then it will be ruled out in this form from the reach of the LHC. Non-resonant physics as well a resonant decays, particularly in the clean dilepton channel, will be some of the first physics to be seen at a new collision energy if it exists. If it is not found this does not mean there are not other beyond the SM process that could exist in nature but would have to be less obvious or at a higher energy scale. The SM has been shown to make very accurate predictions for a host of phenomena yet we know it is not a complete theory. If it holds up within the energy range of the LHC is yet to be seen. 



% add comparison to CMS results!!!!!


