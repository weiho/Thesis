\chapter{Experiment}

This chapter will explore the ATLAS experiment and the LHC which provides proton beams to it.

\section{The Large Hadron Collider}

The Large Hadron Collider (LHC) is the largest and most powerful particle collider in the world with a circumference of $27~km$ and and design centre of mass collision energy of $14~TeV$. During the 2012 run the accelerator was run at a centre of mass energy of $8~TeV$ while providing an integrated luminosity of just above $20~fb^{-1}$ throughout the year to its two general purpose experiments, CMS and ATLAS, that latter of which provided data for this analysis.
However the LHC can not run in isolation to provide beams for its 4 main experiments, instead it is the last and newest accelerator in a chain of accelerators which extract protons from a hydrogen canister with little to no momentum and inject them in to the LHC as a 450 GeV beam.
The proton source is a device called a Duoplasmatron which injects hydrogen gas in to a strong electric field striping electrons from their nuclei. The remaining protons are injected in to Linac 2, a linear accelerator which accelerates them to an energy of $50~MeV$. The BOOSTER or Proton Synchrotron Booster (PBS) comes next in the chain and accelerates protons from $50~MeV$ to $1.4~GeV$ to be injected in to the main Proton Synchrotron (PS). The PS accelerates protons up to an energy of $25~GeV$ and again injects them in to another accelerator, the Super Proton Synchrotron (SPS). The SPS is the final stage before injection in to the LHC ring and pushes protons to an energy of $450~GeV$. Protons from the SPS then get injected in to the LHC in both counter revolving directions and accelerated to their final collision energy. For the data used in the analysis that follows (the 2012 data run) the final proton beam energy is $4~TeV$ giving a final centre of mass collision energy of $8~TeV$.






\section{ATLAS Overview - A Toroidal LHC ApparatuS}
{\normalsize - Overview of LHC and the ATLAS detector and its subsystems}

\subsection{Inner Detector}
\subsection{Calorimeters}
\subsection{Magnet System}
\subsection{Muon Spectrometer}
\subsection{Forward Detectors}


- performance and faults

%\section{The Tracking Detector and Electrons}


%\section{The Calorimeters and Electrons}
