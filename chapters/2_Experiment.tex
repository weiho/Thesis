\chapter{Experiment}

	This chapter will explore the ATLAS experiment in order to explain data specific to this analysis is obtained. First however is a discussion of the Large Hadron Collider which supplies the ATLAS experiment with proton collisions.

\section{The Large Hadron Collider}

	The Large Hadron Collider (LHC) is the largest and most powerful particle collider in the world with a circumference of $27~km$ and and design centre of mass collision energy of $14~TeV$. During the 2012 run the accelerator was run at a centre of mass energy of $8~TeV$ while providing an integrated luminosity of just above $20~fb^{-1}$ throughout the year to its two general purpose experiments, CMS and ATLAS, that latter of which provided data for this analysis.
	However the LHC can not run in isolation to provide beams for its 4 main experiments, instead it is the last and newest accelerator in a chain of accelerators which extract protons from a hydrogen canister with little to no momentum and inject them in to the LHC as a 450 GeV beam.
	The proton source is a device called a Duoplasmatron which injects hydrogen gas in to a strong electric field striping electrons from their nuclei. The remaining protons are injected in to Linac 2, a linear accelerator which accelerates them to an energy of $50~MeV$. The BOOSTER or Proton Synchrotron Booster (PBS) comes next in the chain and accelerates protons from $50~MeV$ to $1.4~GeV$ to be injected in to the main Proton Synchrotron (PS). The PS accelerates protons up to an energy of $25~GeV$ and again injects them in to another accelerator, the Super Proton Synchrotron (SPS). The SPS is the final stage before injection in to the LHC ring and pushes protons to an energy of $450~GeV$. Protons from the SPS then get injected in to the LHC in both counter revolving directions and accelerated to their final collision energy. For the data used in the analysis that follows (the 2012 data run) the final proton beam energy is $4~TeV$ giving a final centre of mass collision energy of $8~TeV$.

	The LHC itself is built in the same tunnel used by the Large Lepton-Positron (LEP) collider. Based at CERN (Centre of European Nuclear Research) the $27~km$ tunnel is between $50$ to $175~m$ underground and like CERN itself crosses the French-Swiss border just outside Geneva. Construction of the LHC started in 2001 after the LEP collider was decommissioned dismantled with excavation of the caverns for the LHC's four main experiments starting slightly before in 1998. 
	The LHC is a synchrotron machine requiring 1,232 super-conducting Niobium-Titanium dipole magnets each providing an $8.33~T$ magnetic field to direct the proton beams around its loop and an additional 392 quadrupole magnets of the same type to focus the beams for the collision points. The super conducting magnets operate at $1.9~K$ with the whole accelerator requiring 96 tonnes of liquid helium to remain cooled.\\

	-protons per bunch\\
	-number of bunches\\
	-luminosity (design and running for all of these)\\

	Four collision points exist around its circumference providing events to the four experiments; ATLAS (A Toroidal LHC Apparatus), CMS (Compact Muon Solenoid), ALICE (A Large Ion Collider Experiment) and LHCb (Large Hadron Collider beauty). ATLAS and CMS are both general purpose experiments designed to look for a variety of physics. ALICE is designed specifically to study quark-gluon plasma in heavy ion collisions scheduled for the end of each LHC run period while LHCb looks for beauty mesons in a search for CP-violation.
	There are also three additional LHC detectors in various stages of deployment without their own collision points; TOTEM (Total Elastic and diffractive cross section Measurement), LHCf (LHC forward) and MoEDAL (Monopole and Exotics Detector at the LHC) which measure separate beam properties. TOTEM shares CMS's collision point aiming to measure the proton cross-section very accurately while LHCf shares ATLAS's collision point measuring the very forward region of collision with the hope of investigating the source of ultra-high-energy cosmic rays. MoEDAL shares a cavern with LHCb and targeted to search for magnetic monopoles and other highly ionising stable massive particles.




\section{ATLAS - A Toroidal LHC Apparatus}

	- image of whole detector

	The ATLAS detector sits $100~m$ underground just over the road from the main CERN site and at $45~m$ long, $25~m$ in diameter and weighing over 7,000 tons is one of largest and most complex particle physics experiments in the world. The Detector it's self can be divided in to four main subsystems and from the interaction point out they are; the Inner Detector or tracking detector, the Calorimeters both Electro-Magnetic (EM) and Hadronic, the Magnet system and the Muon Spectrometer. There is also a small set of forward detectors for accurately measuring the integrated luminosity provided to ATLAS by the LHC. 

	\begin{equation}
		\eta~=~-\ln[\tan(\frac{\theta}{2})]
	\end{equation}

	- detector coordinates system.\\
	- sub detector coverage and diagram.\\

	Broadly the detector is also divided in to the barrel region (cylinder surrounding the interaction point) and endcap region (circles covering the ends of the barrel region) which use slightly different configurations and technology in order to cover a full range in $\eta$.
	Following is a description of each main subsystem while focusing particularly on both the Inner Detector and EM Calorimeters as these are the important systems in identification of electrons used for this analysis.

	\subsection{Inner Detector}

		-good diagram

		The Inner detector is ATLAS's main tracking detector fitted as close to the collision point as possible. 


		\subsubsection*{Pixel detector} 

		sadfasdcasdc



		\subsubsection*{Semiconductor Tracker}

		asdasdcasdcsdc




		\subsubsection*{Transition Radiation Tracker}



		asdcsdcasdc


	\subsection{Calorimeters}

		-good diagram

		-EM (barrel, endcap)
		-Hadronic (barrel, endcap)


	\subsection{Magnet System}

		- Solenoid magnet 
		- Toroid magnet 


	\subsection{Muon Spectrometer}

		-good diagram


	\subsection{Forward Detectors}


	- performance and faults?

%\section{The Tracking Detector and Electrons}


%\section{The Calorimeters and Electrons}
