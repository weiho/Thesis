\chapter{Reconstruction}
{\normalsize - Reconstruction of electron objects and selection of ``good'' candidates.}

Reconstruction is the process and algorithms that attempt to reform information about collision events and their decay products from detector signals. This process is done at several points in the ATLAS analysis procedure. First partial reconstruction of RoI's is done at the Level-2 trigger while a mostly full detector reconstruction is done at the EF. After the data has been permanently stored full reconstruction of all possible signatures in each event as well as whole event variables can be completed if it failed to finish live during the trigger decision.

The other main source of reconstruction is done in a process called reprocessing. After data has been stored updates to sub-detector calibrations and optimisations can take place and so reconstruction of entire data sets takes place to update variables to more accurate measurements. 


Bellow will mainly be a discussion of the reconstruction of electron (and related photon) objects as these are the decay products searched for in this analysis.



\section{Electron Reconstruction}
\label{sec:ReconElec}


	medium loose tight requirements
	trigger variables from paper table.
	variables used in analysis.
	



\section{Data Formats and Software}